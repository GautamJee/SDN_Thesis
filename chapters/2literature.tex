\chapter{Literature Survey}

A starter handle of SDN was required to get hypotheses and details that assume significant jobs in the system. The paper underneath was critical for looking for data on the changed networking frameworks that were being used and how a product characterised arrange was distinctive in contrast with the traditional network design.

Distributed in 2014, the paper ''A survey and a layered taxonomy of software-defined networking'' \cite{taxonomy2014} advances a framework that groups distributed research works and exposes the best possible way to seek after research. The paper depicted that there is a need to arrange and look at SDN Controllers yet from an alternate point of view. As opposed to simply estimating crude execution, the situation must be considered in which the working framework, runs just as what the system's goal is.

Also, the paper titled `` Software-defined Networking (SDN): a survey ''\cite{benzekki2016survey} and published in 2016, apart from the description about SDN, its controllers and architecture, it provides details about the issues prevailing in adopting SDN by existing networking giants.

Further research was done to see how to look at changed network working frameworks. There are many open-source organize controllers accessible. Be that as it may, to locate the best controller, first, there is a need to set up the significance of "the best controller" or whether there is any importance to such an idea.

A paper distributed in 2014, titled ''Software-Defined Networking: A Comprehensive Survey'' \cite{Survey2014} played out an investigation of the hardware utilized in the software-defined network design. The network utilizes basic switch like gadgets instead of completely useful routers. While data forwarding components got idiotic, the general effectiveness of network traffic altogether improved as routers were adequately straight forwarding packets. The control plane components are presently spoken to by a solitary substance called the "Controller" or Network Operating System. Further outsider applications are likewise permitted to be actualized on the network logic with authorisation from the controller to give a few highlights explicit to the working framework and are a lot simpler to create and convey.

So as to analyze controllers, first, there is a requirement for a lot of measurements and apparatuses with which the controller's performance can be recognized and further grouped based on network situations.

Published in 2016, the paper ``SDN Controllers: A Comparative Study'' \cite{adaptiveroute2006} tested the performance of different controllers using CBench. It concluded that controllers coded by the C-language gave the best performance and when all is said, the Java-based distributed controller OpenDayLight performed well.

Published in 2015, the paper ``On the performance of SDN controllers: A reality check'' \cite{realitycheck}  also tested throughput and latency using CBench for 5 different controllers namely Pox, Nox, Ryu, Floodlight, and Beacon. Their results concluded that Beacon has a better performance. Also, they found that Ryu is easy to learn and highly accessible.

Published in 2018, the paper ``Performance Analysis of POX and Ryu with Different SDN Topologies’’ shows a comparative study between two python-based controllers and tests them under different network topologies. It concludes that Ryu outperforms Pox in terms of throughput and latency.

Now the measurements that were generally used to look at SDN controllers were comprehended, there is a need to discover appropriate instruments and experiments. The following paper provides information as a part of collecting a choice of tools for our analysis.

Published in 2014, the paper ``Using Mininet for Emulation and Prototyping Software Defined Networks'' \cite{mininet2014} is about SDN controller emulation tools like Minine and related techniques. A realistic virtual network that too running on real kernel along with switches and application codes can be created easily on a given machine like native, cloud, or virtual by the use of Mininet.

Published in 2018, the paper ``Benchmarking Methodology for Software-Defined Networking (SDN) Controller Performance'' \cite{rfc8456} characterizes a lot of measurements and comparing systems for benchmarking the control-plane execution of Software-Defined Networking (SDN) Controllers.

Published in 2019, the paper titled ``SDN Controllers: Benchmarking \& Performance Evaluation'' \cite{zhu2019sdn} not only show results on comparative performances of various SDN Controllers but also provide a comparative study on multiple tools like CBench. It also includes the CPU Utilisation Analysis, where OFNet is used to send bulk packets.

Published in 2012, the paper ``A Flexible Open-Flow Controller Benchmark’’ \cite{flexible} proposes changes to the OpenFlow protocol as they discovered inherent CPU performance bottlenecks. It also shows how multiple instances of CBench is used to send packets and measure CPU utilisation.

Published in 2014, the paper titled ``OFCProbe: A platform-independent tool for OpenFlow controller analysis’’ \cite{ofcprobe} shows both OFCProbe and OFCBench being used as a CPU and RAM utilisation monitor and finding bottlenecks for the controller. However, both only works with Ryu, Floodlight, and Nox. It implements the CPU and RAM monitors at the host machine using SMTP messages.

\section*{Summary}
The writing review included distributions identified with either clarification of basics and ideas of programming characterized systems administration, or research demonstrating various uses of controllers in programming characterized organizing dependent on true situations. A few papers likewise talked about strategies for looking at controllers dependent on execution and other applicable elements. Various measurements utilized for controller execution correlation were utilized, and the importance of those measurements was additionally talked about. The survey clarifies that there is a need for a comparison of different controllers. Previous papers mostly published performance results of controllers based on network size, throughput, and latency. Those who published about system utilisation only include information about CPU and RAM and excluded information about cache memory, etc. which are also major factors in performance. Furthermore, the tools necessary for sending bulk packets to the controller operating system, such as OFNet, is now a dead project and is not available. OFCBench and OFCProbe use SMTP messages to get data from the controller and are limited to a few Benchmarks, whose data is less accurate.