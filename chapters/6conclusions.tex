\chapter{Conclusion \& Future Works} 

The experiment was successfully performed on two different SDN controllers (Ryu and Pox) as planned under heavy traffic and with the latest stable releases of both controllers. The results provide a clear picture of their system-level performances. It also shows system performance analysis of SDN controllers is a challenging task.  It can be observed that Ryu utilized less CPU than Pox as the former does not support multi-threading while the later does. The utilization of first level cache by both the controllers was reduced and scored more than 70\%. However notably, both controllers efficiently use other higher-level caches and branch predictors. Further is also observed that both controllers have a decent score of instructions per cycle, i.e. greater than 1, which signifies the pipelining is used efficiently. Pox even has a better efficiently coded initializing module(s) than of Ryu's, as seen from the lesser branch mispredictions and higher IPC during the starting time.

This experiment also makes itself different and competitive from others as it was performed under heavy traffic and analyzed on the latest releases of both controllers. Since OFNet is no longer available \cite{}, this experiment successfully shows the efficient mutual use of CBench and Mininet to send random packets.

The use of the perf tool in the experiment shows how this tool is capable of measuring system-level metrics right from the server itself, rather than using OFCBench or OFCProbe which were limited to Floodlight, Ryu and Nox and also ran on host machines, sending SMTP messages to get data from server.

In future works, some of the limitations faced by this research could be looked into, such as the lack of hardware and time. This experiment also displayed how system bottlenecks identification can be made, which will help the developers to improve SDN Controllers running on a server and also identify which system will be better for a given controller or which controller will be better for a given system. This experiment also lays down information to help developers know about the programming lags. It thus will help them write more optimized modules of SDN Controllers to utilize the system better. This type of profiling will also encourage to use the compilers like LLVM, Numba, etc. to develop or improve controllers.

Unlike OMNET++, a popular network application used to simulate computer networks has been a part of SPEC CPU Benchmark Suite since 2006; one of the various SDN Controllers also have a higher possibility, viewing the increasing popularity among multiple networks and hardware corporate giants, that controller will be developed using the profiling as done in this experiment.