\chapter*{Annexure - B}
\addcontentsline{toc}{chapter}{Annexure-B}

\section*{Creating virtual networks}

\begin{itemize}

    \item Mininet Configuration

    \begin{itemize}
    
        \item To create virtual network and attach to the controller\\
        Example: \\
        \emph{sudo mn --topo tree,depth=5,fanout=3 --controller=remote,ip=192.168.5.12,port=6633}
        \item NOTE: Default port used by Ryu is 6653 whereas Pox uses 6633.
        \item To open terminal for any host use xterm
        Example: \emph{xterm h1 h2 h10}
    \end{itemize}
    
    \item CBench Configuration
        
    \begin{itemize}
        \item Command to run CBench\\
        Example: \emph{cbench -c 192.168.5.12 -p 6633 -l 100 -s 32 -M 100000 -t}\\
        This is to run the Cbench tool against 32 swtiches (-s) with 100000 hosts (-M) per switch in throughput mode (-t) or latency mode (-l) at localhost controller ip (-c) on port 6633 (-p) 
    \end{itemize}
    
\end{itemize}

\section*{Implementation of Controllers}
\begin{itemize}
\item Ryu \\
Usage: \emph{ryu-manager --verbose ryu/app/simple\_switch\_13.py}
\end{itemize}
\begin{itemize}
\item Pox \\
Usage: \emph{pox.py samples.pretty\_log forwarding.l2\_learning}
\end{itemize}

\section*{Perf Usage}
    \begin{itemize}
    
        \item Syntax\\
        \emph{sudo perf stat -e $< events\ list >$ } sleep $< delay > < process\ ids >$ \\
        \item Example\\
        \emph{sudo perf stat -e task-clock,br\_misp\_retired.conditional,branch-instructions,branch-misses, \\ l2\_rqsts.miss,l2\_rqsts.references,bus-cycles,cache-references,cpu-cycles,instructions  sleep 30 14880,14883,14882,14884}\\
\end{itemize}
