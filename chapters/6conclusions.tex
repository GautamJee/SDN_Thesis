\chapter{Conclusion \& Future Works} 

The tests were successfully performed on three SDN controllers (OpenDayLight, Floodlight and OpenVSwitch) as planned and the results were recorded and analyzed. Moreover, the results provide us with a clear picture of how to classify controller based on network scenario.From the test results, it is observed that OpenDayLight performs better in throughput mode, in less complex networks. This indicates that network scenarios such as that of a corporate building, where there are very few autonomous sub-networks, the switch to host ratio would be fairly low and traffic would be at peak values during the working hours thus causing a heavy request flow. Such scenarios are ideal for OpenDayLight.Floodlight controller shows poor performance in less complicated networks. But once the network complexity crosses a level, the controller maintains a fairly stable performance comparable to its competitors. Network scenarios such as large college campus networks or military base networks where the switch to host ratio is fairly large would prove to be in this controller’s comfort zone.Lastly, OpenVSwitch is a relatively simple controller that does not boast too many features but provides a constant performance throughout the graph. There are no specific network topologies where this controller shines. But OpenVSwitch is an ideal choice as a backup controller, in case the main controller fails, and the reliable performance of basic features is needed until the main controller is debugged and brought back online. 

In future works, some of the limitations faced by this research could be looked into such as the lack of hardware and time. The Switch to Host ratio range taken was too small to generate highly accurate data. In order to increase the range of network topologies, better tools are required to deploy controller operating systems. Machines with more RAM and processing power would provide us with larger network configurations to work with and the results would be more significant.

A network configuration consisting of 100000 hosts per switch must be deployed and tested to accurately compare the benchmarking techniques with other publications. Further work can be done in the area of random traffic network configurations. Here, the network configuration and traffic were simulated to stay constant. Although it does provide fair grounds to compare controllers, the features of some controllers can allow them to outperform others in situations of chaos and unpredictable traffic. The aspect of network security features deployed by controllers can be considered as a performance factor. Tests can be done to check how resilient controllers are to external attacks or abrupt failures within the network while traffic is flowing at an expected rate. Physical metrics such as power consumption and memory usage during different scenarios such as high traffic, low traffic, during external attack or device failures also can provide different perspectives to compare SDN controllers.
