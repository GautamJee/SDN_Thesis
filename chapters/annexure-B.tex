\chapter*{Annexure - B}
\addcontentsline{toc}{chapter}{Annexure-B}

\section*{Implementation of OpenVSwitch}

\begin{itemize}

    \item Mininet Configuration
        
    \begin{itemize}
    
        \item Command from mininet installation folder\\
        \emph{sudo python miniedit.py}
        
        \item Select controller type as OpenVSwitch and press Run. The controller will listen for switches to port 6633 on localhost ip address
            
    \end{itemize}
    
    \item Cbench Configuration
        
    \begin{itemize}
    
        \item Command from Cbench installation folder\\
        \emph{./cbench -c localhost -p 6633 -s 16 -M 1 -t}\\
        to run the Cbench tool against 16 swtiches (-s) with 1 host (-M) per switch in throughput mode (-t) or latency mode (-l)
        at localhost controller ip (-c) on port 6633 (-p) 
        
        
    \end{itemize}
    
\end{itemize}

\section*{Implementation of OpenDayLight}

\begin{itemize}
    
    \item Vagrant Machine
    
    \begin{itemize}
    
        \item Command to boot up vagrant machine from installation folder \\
        \emph{vagrant up}
        \item Command to ssh into the vagrant machine\\\emph{vagrant ssh}
        \item Command to exit from Vagrant ssh \\ \emph{exit}
        \item Command to gracefully shutdown Vagrant machine \\ \emph{vagrant halt} 
        
    \end{itemize}
    
    \item WCbench Configuration within Vagrant
    
    \begin{itemize}
    
        \item Command to start OpenDayLight controller from location "/wcbench"\\ \emph{./wcbench -o}
        
        \item Command from location "/wcbench/cbench" to benchmark the controller\\
        \emph{./cbench -c localhost -s 16 -M 1 -t}\\
        
        to run the Cbench tool against 16 swtiches (-s) with 1 host (-M) per switch in throughput mode (-t) or latency mode (-l)
        at localhost controller IP address (-c)
        
        \item Command to stop OpenDayLight from location "/wcbench" \\ 
        \emph{./wcbench -k}
        
    \end{itemize}
    
\end{itemize}

\section*{Implementation of Floodlight}

\begin{itemize}
    
    \item Floodlight Machine 
    
    \begin{itemize}
    
        \item Open Virtual Box
        \item Load and Boot Up Floodlight Image
        \item Command to start floodlight controller within location "/floodlight"\\
        \emph{java -jar target/floodlight.jar}
        \item open separate terminal and run ifconfig to note down IP Address
        
    \end{itemize}
    
    \item Cbench Configuration
    
    \begin{itemize}
    
        \item Command from location "/wcbench/cbench" to benchmark the controller\\
        \emph{./cbench -c (IP Adress of floodlight controller) -p 6653 -s 16 -M 1 -t}\\
        
        to run the Cbench tool against 16 swtiches (-s) with 1 host (-M) per switch in throughput mode (-t) or latency mode (-l)
        at localhost controller IP address (-c)  on port 6653  
    
        \end{itemize}
    
    \item Perf Usage
    
    \begin{itemize}
    
        \item Syntax\\
        \emph{sudo perf stat -e <events list> } \\
    \end{itemize}
    
    \begin{itemize}
        \item Example\\
        \emph{sudo perf stat -e task-clock,br\_misp\_retired.conditional,branch-instructions,branch-misses,l2\_rqsts.miss,l2\_rqsts.references,bus-cycles,cache-references,cpu-cycles,instructions  sleep 30 14880,14883,14882,14884}\\
    \end{itemize}
\end{itemize}
